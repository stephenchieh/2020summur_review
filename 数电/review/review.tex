\documentclass{article}
\usepackage[UTF8]{ctex}%支持中文
%\usepackage{ctex}
\usepackage{booktabs}%绘制三线表
\usepackage{float}%禁止浮动
\usepackage[a4paper]{geometry}%纸张大小和页边距

\usepackage{titlesec}%修改Latex默认section,subsection样式
\titleformat*{\section}{\Large\centering\bfseries}
%\titleformat{\section}{\centering\Large\bfseries}{\thesection\}{1em}{}

\geometry{left=2.5cm,right=2.5cm,top=2.5cm,bottom=2.5cm}%页边距设置
\title{数电期末复习}
\author{191220090沈天杰}

\begin{document}
	\maketitle
	%目录
	%\newpage
	\tableofcontents
	\newpage
	\section{数制与编码}
	\section{组合逻辑设计原理(布尔代数)}
	\section{组合逻辑电路分析与设计}
	\textbf{扇入}指在特定的逻辑系列中,门电路所具有的输入端的数目。\textbf{扇出}指逻辑门电路在不超出其最坏情况负载规格的条件下,能够驱动输入端个数。扇出是必须要计算的。
	\subsection{组合逻辑器件}
	\subsubsection{译码器}
	是输入n位二进制码,输出$2^n$中取1码的器件。一般的,对于每一种输入可能,只有一个输出信号有效。例如,对于2-4译码器74HC139,其输出是高电平有效的,也即当使能端有效时,将输入的n位二进制数化为10进制数p后,输出中的第p位为1,其余位均为0。当使能端无效时,所有输出均为0。而对于3-8译码器74HC138,其输出为低电平有效的。它有3个使能端,当且仅当3个使能端均有效时,输入才有效。
	\subsubsection{编码器}
	对每个输入信号分配一个唯一的二进制编码的器件称为编码器。是将译码器的输入变为输出,输出变为输入的结果。\par
	优先级编码器:对于一个编码器,在任何一个特定的时刻,可能出现多于一个输入同时有效。如果出现多个输入同时有效,输出按输入优先级编码的编码器称为优先级编码器。例如,3位优先级编码器74HC148,其编码优先级为I7-I0,也就是说如果I5和I4同时有效,输出为I5的编码结果。另外,其输入是低电平有效的。它还有两个输出,GS和EO。GS有效时说明此时有一个或多于一个输入有效,而EO有效时说明此时无有效输入。
	\subsubsection{数据选择器}
	是一种有$2^n$个数据输入,n个选择输入,1个输出的器件。首先,n个选择输入进入输出为高电平的n-$2^n$译码器。然后再将各位数据输入与该译码器的各位输出相与,最后,上述结果中只要有一位为1,输出就为1。\par
	用具有n个选择变量的多路选择器实现n个变量的函数:\par
	将函数的n个变量依次连接到n个选择变量端,并将函数表示成最小项之和的形式。若函数表达式中包含最小项mi(例如,3输入中的A0A1A2代表m7),则相应输入的Di接1,否则Di接0。\par
	用具有(n-1)个选择变量的多路选择器实现n个变量的函数:\par
	从函数的n个变量中任选(n-1)个作为选择变量,一个作为数据输入,并根据各选择变量的输出结果来确定在每种选择的情况下,对应数据变量输入的值Di。这个值只能是0,1,X或$\bar{X}$。若在某种选择下,不管剩余变量的值为0还是1输出均为1,则该选择对应的Di接1,以此类推。\par
	\subsubsection{半加器\&全加器}
	半加器:是指仅考虑加数和被加数的运算部件。
	全加器:考虑加数、被加数和相邻低位的进位的运算部件。半加器和全加器的输出均包括运算结果和向下一位的进位。如:四位先行进位加法器74HC283,可用于进行两个四位二进制数的加法。它有八个分别表示两个四位二进制数的输入,四位加法输出和一位进位输出。
	
	\subsubsection{算术逻辑单元}
	能够对2个n位的操作数进行若干不同的算术和逻辑操作的器件,由一组功能选择输入来指定要执行的操作。对于一组特定的功能选择输入,可以对操作数进行不同的运算。
	\subsubsection{三态门}
	有一个使能端的缓冲器或反相器。当使能端无效的时候,直接将输入信号阻断;当使能端有效的时候,起到反相器或缓冲器的作用。
	\subsubsection{数据总线}
	是将若干条数据线合成一条的表示法,用[0:n]表示,其含义为序号从0到(n-1)的n根数据线。
	奇偶校验发生器:是检验输入的信号中有奇数个还是偶数个1的器件。例如:74HC280,有9个输入和2个输出EVEN和ODD,当输入包含偶数个1时,EVEN输出为1。
	\subsection{组合逻辑电路分析}
	给定原理图:分析功能,化简电路,分析冒险等。
	\subsection{组合逻辑电路设计}
	给定功能描述:基于约束设计出符合要求的电路。
	
	\section{时序逻辑电路分析与设计}
	\subsubsection{同步时序电路分析}
	同步时序电路的分析步骤:
	\begin{enumerate}
		\item 写出每个触发器的激励输入方程(输入和现态的函数)。
		\item 将激励方程代入到触发器的特征方程,得到转移方程。
		\item 写出电路的输出方程。
		\item 根据转移方程,列出所有输入、现态和输出、次态的组合,得到状态转移(输出)表。
		\item 根据状态转移表画出状态转移图。
		\item 说明电路功能及其能否自启动。
	\end{enumerate}

	\subsubsection{同步时序电路设计}
	同步时序电路的设计步骤:\\
	1、逻辑抽象:根据逻辑问题设定状态,构建原始状态转换图(表)。\\
	2、状态化简:去除冗余状态,得到最小状态表。\\
	3、状态编码:给状态分配一个二进制编码。\\
	4、选择触发器:按照电路的需求,选择最合适的触发器类型。\\
	5、导出方程:根据编码状态转换表和触发器特征方程,导出输出方程和激励方程。\\
	6、画电路图:根据输出方程和激励方程画出逻辑电路图。\\
	7、检查自启动:检查电路能否自启动。\par
	状态化简:是指要获得一个最小化的状态表,这个表不仅能正确地反映设计的全部要求,而且状态的数目最少。\par
	等价状态:设状态Si和Sj是状态表中的两个状态,如果对于所有可能的输入序列,分别从状态Si和Sj状态出发,所得到的输出响应(次态和输出)序列完全相同,则状态Si和Sj是等价的。等价状态可以合并,状态化简的目的是合并所有等价状态。\par 
	蕴含表化简:是一种进行等价状态判定,进行状态化简的方式。\par
	蕴含表化简的步骤为:
	\begin{enumerate}
		\item 构建一个直角三角形阶梯表,两直角边的网格数相同,用状态名进行顺序标注。网格数等于原始状态表中的状态数少 1。纵坐标从上到下标注,缺少第一个状态;横坐标从左到右标注,缺少最后一个状态。
		\item 横纵坐标交汇的每个方格代表一个状态对。若方格对应的两状态对能确定等价,则方格内画勾;能确定不等价,则方格内画叉。若与其它状态对有关,则填写相关状态对。
		\item 进行关联比较,确定等价状态对。
	\end{enumerate}	
	状态编码:赋予状态一个唯一的二进制编码的过程。1位二进制编码需要用1个触发器表示。
	
%	\begin{enumerate}
	相邻状态分配法:是一种状态编码的规则。包括以下几条:
	\begin{itemize}		
		\item 规则1,次态相同,现态相邻。对于给定的输入,具有相同次态的现态应该分配逻辑相邻的编码。
		\item 规则2,同一现态,次态相邻。对于一个现态,在逻辑相邻输入下的次态,应该分配逻辑相邻的编码。
		\item 规则3,输出规则。在相同输入情况下,具有相同输出的状态应该分配相邻的编码。
		\item 规则4,已经利用规则1、2、3确定分配相邻编码的状态对,其次态对应该分配相邻的编码。
		\item 规则5,在进行状态分配时,先找到在化简的状态表中作为次态次数最多的状态,分配全0的编码,然后利用规则1、2、3、4进行状态分配。尽可能的满足所有的规则。
	\end{itemize}
		
	%\end{enumerate}
	
	
	\subsubsection{同步时序电路实例}
	\section{思考题题型}
	
\end{document}