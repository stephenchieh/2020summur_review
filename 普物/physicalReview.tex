\documentclass{article}
\usepackage[UTF8]{ctex}
\usepackage{amsmath,mathtools}
\usepackage{color}
\usepackage{float}
\usepackage{esint}%积分符号
\newcommand{\point}[1]{$\color{blue}{\text{#1}}$}
\setlength{\parindent}{0pt}%不缩进
\usepackage[a4paper]{geometry}%纸张大小和页边距
\geometry{top=2.5cm,bottom=2.5cm}%页边距设置
\usepackage{titlesec}%修改Latex默认section,subsection样式
\titleformat*{\section}{\Large\centering\bfseries}

\title{普物复习}
\author{191220090 沈天杰}
\begin{document}
    \maketitle
    {\centering\tableofcontents}
    \newpage
    \section{力学}
    \subsection{牛顿定律}
    \begin{enumerate}
        \item 第三定律中的力只是宏观的经典概念,超出这个范围并不好用。
        \item 考虑到相互作用传递低于光速后的图像。
    \end{enumerate}
    \subsubsection{\point{考题}}
    主要是受力分析。惯性力。\\
    相对运动,恒力作用下的直线运动曲线运动,滑轮模型,摩擦力,变力作用下的直线运动。
    \[
      a=\frac{dv}{dt}=\frac{dv}{dx}v  
    \]
    \subsection{动能,机械能}
    若定义无限远处为引力势能零点
    \[
        E_p=-\frac{GmM}{r}
    \]
    一对非保守内力(耗散力)做负功,使系统动能减少。\\
    机械能守恒(质点在只受保守力场的作用时)
    \[
      E_K+V_A=const  
    \]
    其中$E_K$为质点动能,$V$为质点在保守力场中的势能。
    \subsection{动量,角动量,守恒律}
    \subsubsection{动量}
    \begin{enumerate}
        \item 动量定理只适用于惯性系,对非惯性系。。。 
        \item 在牛顿力学的理论体系中,动量守恒定律是牛顿定律的推论。但动量守恒定律是更普遍、更基本的定律,它在宏观和微观领域、低速和高速范围均适用
    \end{enumerate}
    \point{经典例题}
    \begin{figure}[H]
        \centering
        \includegraphics[width=.65\textwidth]{figure/pin.png}
    \end{figure}
    子弹质量忽略不计
    \subsection{角动量}
    有心力:角动量守恒$L=J \omega= Mt$
    \begin{figure}[H]
        \centering
        \includegraphics[width=.65\textwidth]{figure/omega.png}
    \end{figure}
    \subsection{刚体}
    转动惯量
    \begin{equation*}
        J=
        \begin{cases}
            mR^2 & \text{圆环}\\
            \frac{1}{2}mR^2& \text{圆环直径 ;圆盘}\\
            \frac{2}{5}mR^2& \text{球体}\\
            \frac{2}{3}mR^2& \text{球壳}\\
            \frac{1}{3}mR^2& \text{细棒(端点)}\\
            \frac{1}{12}mR^2& \text{细棒(中心轴)}\\
        \end{cases}
    \end{equation*}
    \begin{figure}[H]s
        \centering
        \includegraphics[width=.65\textwidth]{figure/fill.png}
    \end{figure}
    \begin{equation*}
        \begin{cases}
            r \sim \theta \\
            v \sim \omega \\
            a \sim \alpha \\
            F \sim M \\
            m \sim J \\
        \end{cases}
    \end{equation*}
    对滑轮
    \[
    M=FR=J \alpha    
    \]
    平行轴定理
    \[
        J=J_c+md^2  
    \]
    \section{气体运动论}
    理想气体物态方程
    \[
    pV=vRT    
    \]
    \[
      p=\frac{m}{VM}RT=\frac{NM_0}{VN_Am_0}RT=n\frac{R}{N_A}T=nk_BT  
    \]
    玻耳兹曼常数$k_B=\frac{R}{N_A}=1.39 \times 10^{-23} (J/K)$\\
    麦克斯韦速率分布律:
    \[
      f(v)=4\pi (\frac{m}{2\pi kT})^{3/2}e^{-mv^2/2kT}\cdot v^2  
    \]
    其中m为单个粒子质量,k为玻尔兹曼常数\\
    方均根速率
    \[
    \sqrt{\overline{v^2}}=\sqrt{\frac{\sum\limits_{i=1}^n v_i^2N_i}{N}}    
    \]
    最概然速率$v_p$:出现次数最多的速率
    \begin{align*}
             v_p&=\sqrt{\frac{2kT}{m}}    \\
             \overline{v} &\approx 1.1v_p \\
             \sqrt{\overline{v^2}}&=\sqrt{\frac{3kT}{m}} \approx 1.2v_p 
    \end{align*}
    平均平动动能
    \[
    \overline{\varepsilon_k}=\frac{1}{2}m\overline{v^2}=\frac{3}{2}k_BT
    \]
    自由度i
    \begin{table}[H]
        \centering
        \setlength{\tabcolsep}{16mm}{
        \begin{tabular}{|c|c|}
        \hline
        单原子分子 & 3 \\ \hline
        双原子分子 & 5 \\ \hline
        多原子分子 & 6 \\ \hline
        \end{tabular}}
        \end{table}
    内能$U=v\frac{i}{2}RT$
    \section{热力学}
    \begin{itemize}
        \item 热力学第零定律定义了温度这一物理量,指出了相互接触的两个系统,热流的方向。
        \item 热力学第一定律指出内能这一物理量的存在,并且与系统整体运动的动能和系统与环境相互作用的势能是不同的,区分出热与功的转换。
        \item 热力学第二定律涉及的物理量是温度和熵。熵是研究不可逆过程引入的物理量,表征系统透过热力学过程向外界最多可以做多少热力学功。
        \item 热力学第三定律认为,不可能透过有限过程使系统冷却到绝对零度。
    \end{itemize}

    \begin{enumerate}
        \item 热力学第零定律:在不受外界影响的情况下,只要A和B同时与C处于热平衡,即使A和B没有热接触,他们仍然处于热平衡状态。这个定律说明,互相处于热平衡的物体之间必然具有相等的温度。
        \item 热力学第一定律:能量守恒定律对非孤立系统的扩展。此时能量可以以功W或热量Q的形式传入或传出系统。热力学第一定律表达式为:${\displaystyle \Delta E_{\text{int}}\ =E_{\text{int,f}}-E_{\text{int,i}}=Q-W}$
        \item 热力学第二定律:孤立系统熵(失序)不会减少──简言之,热不能自发的从冷处转到热处,而不引起其他变化。任何高温的物体在不受热的情况下,都会逐渐冷却。这条定律说明第二类永动机不可能制造成功。热力学第二定律也可表示为熵增原理:${\displaystyle \Delta S\geq 0}\Delta S\geq 0$。
        \item 热力学第三定律:完整晶体于绝对温度零度时(即摄氏-273.15度),熵增为零。
    \end{enumerate}
    \subsection{热一}
    等容变化,等压变化,绝热变化
    \begin{align*}
        C_{V,m}&=\frac{i}{2}R\\
        C_{p,m}&=\frac{i+2}{2}R
    \end{align*}
    绝热过程物态方程
    \[
        pV^\gamma=V^{\gamma-1}T=p^{\gamma-1}T^{-\gamma}=C
    \]
    \subsection{热二}
    热机效率
    \[
    \eta=1-\frac{Q_2}{Q_1}=1-\frac{T_2}{T_1}    
    \]
    熵(可逆过程公式)
    \[
     dS=\frac{dQ}{T}   
    \]
    可逆的绝热过程是等熵过程。等熵过程的对立面是等温过程,在等温过程中,最大限度的热量被转移到了外界,使得系统温度恒定如常。由于在热力学中,温度与熵是一组共轭变量,等温过程和等熵过程也可以视为“共轭”的一对过程。
    
    \section{静止电荷的电场}
    \noindent基本物理量: 电荷\quad电场强度\quad电势\footnote{电势梯度与场强}\\
    基本定律:电荷守恒定律,库仑定律,场强叠加原理,高斯定律,安培环路定理
    \subsection{电场、电荷\quad库仑定律}
    \[
        \vec{F_{12}}=\frac{1}{4\pi\epsilon_0}\frac{q_1q_2}{r_{12}^2}(\frac{\vec{r_{12}}}{r_{12}})    
    \]
    其中$\frac{1}{4\pi\epsilon_0}\approx 8.988\times 10^9 N\cdot m^2\cdot C^{-2}$\\
    \point{考题}:库仑定律和力的叠加原理
    \subsection{电场\quad电场强度}
    \noindent电场强度是随位置而变的矢量场
    \[
      \vec{E}=\frac{\vec{F}}{q_0}=\frac{Q}{4\pi\epsilon_0r^2} \hat{r} \quad \textbf{点电荷的场强} 
    \]
    场强叠加原理 离散与连续。\\\\
    \point{考题1}:两电荷连线的中垂面上任意一点P的电场强度\\
    电偶极子(或称电偶极矩):$\vec{p}=q\vec{l}$\\
    电偶极子在其延长线上远场点电场强度$E=\frac{p_e}{2\pi\epsilon_0r^3}$\\
    注意\;1)\;l很小(相对于r) 2)\;方向从负电荷指向正电荷。\\
    点P的电场强度方向与电偶极子相反。电偶极子的电场强度是立方衰减的。\\
    \point{考题2}:无限长均匀带电细棒中垂面上的场强分布\\
    \[
        E_y=\frac{1}{4\pi\epsilon_0}\frac{2\lambda}{a}
    \]
    其中a为待测点到细棒距离。\\
    \point{考题3}:带电圆环轴线上的电场强度。P点离环心的距离为x。\par
    \[
        \vec{E}=\frac{qx}{4\pi\epsilon_0(R^2+x^2)^{3/2}}\hat{i}  
    \]
    当$x\to \infty$电场强度时等同于点电荷\\
    \point{考题4}:带电圆盘轴线上的电场强度。圆盘半径为R,面密度为$\sigma$:\\
    \[
    E_x=\frac{\sigma}{2\epsilon_0}(1-\frac{1}{\sqrt{1+R^2/x^2}})    
    \]
    1)R$\ll$x \; 点电荷 \\
    2)R$\gg$x \; $E_x=\frac{\sigma}{2\epsilon_0}$ \; 相当于无限大板。匀强电场,和距离无关。\\
    \point{考题5}\textbf{无限大面和无限长线上电荷的综合运用(重点)}
    \begin{figure}[H]
        \centering
        \includegraphics[width=.95\textwidth]{figure/inf.png}
    \end{figure}
    \point{总结}:计算场强分布的三种办法:\\
    1。库仑定律+场强叠加原理2。电荷对称性分布:高斯定理3。电势梯度
    \subsection{电场线}
    \noindent电场线:电场线上每一点的切线方向都与该点的场强E方向一致;在与电场强度垂直的单位面积中,所穿过的电场线根数与该处的场强大小成正比。
    \[
    dN\sim EdS \quad E\sim dN/dS    
    \]
    即:场强正比于与其垂直的单位面积内穿过的电力线根数。\\
    性质:
    \begin{enumerate}
        \item 起自正电荷(或无限远),终止于负电荷(或伸向无穷远),但不会在没有电荷的地方中断。(高斯定理)
        \item 静电场的电场线不能形成闭合曲线,无旋场。(环路定理)
        \item 电力线越密的地方,场强越大;电力线越疏的地方,场强越小。
        \item 任何两条电场线不会相交。
        \item 电场线的方向反映正电荷在各点的受力方向,但电场线不是正电荷的运动轨迹。
    \end{enumerate}
    借助电场线,引入\point{电场强度通量}$\psi_E=ES$\\
    对整个曲面积分可求得面积为S的任意曲面E通量
    \[
    \psi_E=\iint_S Ecos\theta dS =\iint_S E\cdot dS
    \]
    S为闭合曲面时,曲面内部穿出E通量为正,外部穿入E通量为负。
    \[
    \psi_E=\oiint_S Ecos\theta dS =\oiint_S E\cdot dS
    \]
    \subsection{高斯定理}
    \noindent以点电荷q为球心的球面的E通量都等于$q/\epsilon_0$\\
    通过电场中任一闭合曲面的总电通量,等于该曲面内包围的所有电荷电量的代数和除以$\epsilon_0$,而与闭合面外的电荷无关。
    \[
      \oint_S \vec{E}\cdot d\vec{S}=\frac{1}{\epsilon_0}\sum_{\text{S内}} q_i  
    \]
    虽然高斯面上的电通量只和内部电荷量有关,但不能说:高斯面上电场只是由内部电荷决定的。高斯面上的电场是由全空间电荷共同决定的。\\
    \point{高斯定理的应用}\\
    适用情况:通常是具有某种对称性的电场--轴对称、球对称、均匀场等。\\
    应用方法:先作对称性分析。\\
    静电场是有源场.
    \subsection{静电场的环路定理、电势(电位)}
    \subsubsection{环路定理}
    场强环路定理——\point{静电场}中,沿任一闭合路径场强的环流等于零.
    \[
      \oint_L \vec{E}\cdot d\vec{l}=0  
    \]
    物理含义:静电场是保守力场(可定义电势能和电势);微分形式即无旋场。
    \subsubsection{电势(电位)}
    电场中a点的电势是描写电势能\footnote{电荷$q_0$在电场中某点a具有的电势能\textbf{等于}电场力将此电荷从参考点移至a点电场力所作的功的负值。}的物理量。
    \[
    V_a=w_a/q_0=-\int_{\text{参}}^a \vec{E}\cdot d\vec{l}    
    \]
    特殊的\;单个点电荷电势为$\frac{q}{4\pi\epsilon_0r}$\\
    电势叠加原理:点电荷电场中一点的电势,等于每一点电荷单独在这一点所产生的电势的\point{代数和}。\\
    \point{考题}:带电q半径R球壳
    \begin{enumerate}
        \item 球壳场强
        \begin{equation*}
            \vec{E}=
            \begin{cases}
                0 & (0\leq r \textless R)\\
                \frac{q}{4\pi\epsilon_0r^2}\hat{r}&(R\leq r \textless \infty)
            \end{cases}
        \end{equation*}
        \item 球壳电势
        \begin{equation*}
            V(r)=
            \begin{cases}
                \frac{q}{4\pi\epsilon_0R} & (0\leq r \textless R)\\
                \frac{q}{4\pi\epsilon_0r} & (R\leq r \textless \infty)
            \end{cases}
        \end{equation*}
    
    
    \end{enumerate}
    计算电势的方法:\;1。先计算场强,然后积分计算\;2。叠加原理。
    \subsection{等势面\quad电场强度与电势梯度的关系}
    等势面性质:
    \begin{itemize}
        \item 等势面与电力线处处正交。
        \item 等势面密集的地方场强大,稀疏的地方场强小。
    \end{itemize}
    由梯度的定义,有$dV=\nabla V \cdot d\vec{l}$\\
    场强方向即梯度逆方向,即电势下降最快的方向。
    \[
      \vec{E}=-\nabla V    
    \]
    其中$\vec{\nabla} \equiv \vec{i}\frac{\partial}{\partial x}+\vec{j}\frac{\partial}{\partial y}+\vec{k}\frac{\partial}{\partial z}$
    接触后等势
    \subsection{静电场中的导体}
    \subsubsection{静电平衡}
    静电感应最终使导体内部场强为0。达到静电平衡的状态。(有电流的电线不是静电平衡)
    \begin{itemize}
        \item 导体是等势体,表面是等势面。
        \item 导体表面的电场强度垂直于导体表面。
    \end{itemize}
    \subsubsection{静电平衡导体的电荷分布与电场}
    静电平衡时导体中的电场特性:
    \begin{itemize}
        \item 导体内部的电场强度处处为零。且导体表面的电场强度垂直于导体的表面。
        \item 导体内部和导体表面处处电势相等,整个导体是个等势体。
    \end{itemize}
    \begin{enumerate}
        \item 对实心导体情况
        \item 对导体空腔情况
        \begin{enumerate}
            \item 空腔内无电荷
            \item 空腔内有电荷
        \end{enumerate}
    \end{enumerate}
    \point{静电屏蔽}:在静电平衡下,空腔导体外面的带电体不会影响空腔内部的电场分布;一个接地的空腔导体,空腔内的带电体对腔外物体不会产生影响。

    \subsection{电容器与电容}
    电容的物理含义:升高单位电势所需电量。\;$C=\frac{q}{U}$\;单位:法拉$F$
    \begin{enumerate}
        \item  孤立导体电容定义为$C=\frac{Q}{V}=4\pi\epsilon_0R$
        \item  平板电容器电容定义为$C=\frac{Q_A}{U_{AB}}=\frac{Q_A}{V_A-V_B}=\frac{\sigma S}{Ed}=\frac{\epsilon_0 S}{d}$\\因为平板间场强为$E=\frac{\sigma}{\epsilon_0}$
    \end{enumerate}
   
    \subsection{静电场的能量}
    能量存储在场中。
    \[
      W=\frac{1}{2}\frac{Q^2}{C}=\frac{1}{2}CU^2=\frac{1}{2}\epsilon E^2Sd=\frac{1}{2}D\cdot E \cdot Sd  
    \]
    \section{恒定电流及其磁场}
    \subsection{恒定电流和导电定律}
    \subsubsection{电流密度}
    电流密度为位置的矢量,该点正电荷移动方向。$I=\iint_S \vec{j}\cdot d\vec{S}$\\
    \subsubsection{电流密度与电场强度关系}
    \[
        \vec{j}=\sigma \vec{E}\quad \sigma \text{为电导率}  
    \]
    它实际上是欧姆定律的微分形式$I=\frac{U}{R}$
    \subsection{磁场和磁感应强度}
    磁感应强度B的大小可以用运动的试探电荷在磁场中的受力来表征。\\
    洛伦兹力$\vec{F}=q_0 \; \vec{v} \times \vec{B}$\\
    \subsection{毕萨定律}
    仿照电场,磁场的研究通过电流元$Idl$进行。
    \[
      d\vec{B}=\frac{\mu_0}{4\pi} \frac{Id \vec{l} \times \vec{r}}{r^3}  
    \]
    其中$\mu$为真空磁导率,$\frac{\mu_0}{4\pi}=10^{-7}T\cdot m/A$\\
    运动电荷激发的磁场
    \[
    \vec{B}=\frac{\mu_0}{4\pi}\frac{q\vec{v} \times \vec{r}}{r^3}\quad \rightarrow \text{平方衰减}    
    \]
    \point{结论}
    \begin{equation*}
        \begin{cases}
            \text{无限长电流磁场:}&B=\frac{\mu_0I}{2\pi R}\\
            \text{圆电流圆心处磁场:}&B=\frac{\mu_0I}{2R}\\
        \end{cases}
    \end{equation*}
    \subsection{磁场的高斯定理和安培环路定理}
    \[
        \oint_L \vec{B}\cdot d\vec{l}=\mu_0\sum\limits_\text{L内}I_i  
    \]
    无限长直圆柱形载流导线内外空间磁场的分布
    \begin{equation*}
        B=
        \begin{cases}
            \frac{\mu_0Ir}{2\pi R^2} & (0\leq r \textless R)\\
            \frac{\mu_0I}{2\pi r}&(R\leq r \textless \infty)
        \end{cases}
    \end{equation*}
    \subsection{带电粒子在磁场中的运动}
    %\subsection{磁场对载流导线的作用}
    \section{电磁感应、麦克斯韦方程组}
    %\subsection{电动势}
    %\subsection{电磁感应和楞次定律}
    \subsection{法拉第电磁感应定律}
    穿过闭合回路所围曲面的磁通量发生变化时,导体回路中产生的感应电动势正比于磁通量变化率的负值
    \[
        \varepsilon_i=-\frac{d\Phi}{dt}    
    \]
    其中$\Phi=\iint_S \vec{B}\cdot d\vec{S}$
    \subsection{感生电动势和动生电动势}
    \subsubsection{感生电动势}
    \[
        \varepsilon_i=-\frac{d\Phi}{dt}=-\frac{d}{dt}\iint_S \vec{B}\cdot d\vec{S}    
    \]
    \begin{equation*}
        \begin{cases}
            \text{回路面积(S)改变:}&\text{磁场不动,导体动--动生电动势}\\
            \text{磁场(B)改变:}&\text{导体不动,磁场变--感生电动势}\\
        \end{cases}
    \end{equation*}
    \subsubsection{动生电动势}
    \[
        \Phi=Blv
    \]
    %\subsection{涡旋电场}
    %\subsection{麦克斯韦方程组}
\end{document}